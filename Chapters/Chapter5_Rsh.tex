\chapter{Conclusions and Future Work}
\label{chap:conclusions}
\lhead{\emph{Conclusions}}
% This chapter should comprise 2-3 pages and enumerate conclusions of this phase of work. In your final report Discussions and Conclusions will form separate chapters and be significantly longer and more detailed.

\section{Discussion}
\label{section:discussion}
% A reflective discussion of some of the problems you encountered during this phase of the project and how that may influence how you proceed with the next phase.

In reflecting on the difficulties encountered in this research phase I find that the only major issue I had was translating research down onto paper. However I believe this problem occurred due to attempting to write this report while doing research concurrently. I changed this approach relatively early during the semester and found progress was much smoother afterwards. 

In regards to implementation work, I think the best way to translate this improvement in work process is to put report writing on the sidelines until I have entire parts of the implementation completed, rather than attempting to write parts of the report while the associated part of the implementation is only partially completed.

A very minor issue encountered during this phase was time management. While I was generally able to find ample time enough to invest into this report, unavoidable time constraints caused by assignments in other college modules often lead to not being able to invest the desired time into this report.

For the implementation phase I believe this will be less of an issue to begin with as I will be doing my best to strictly adhere to the sprint schedule outlined in section \ref{section:implementationplan}. Considering I have a sizable amount of experience working with Android for about 6 years now, both as a hobby and a job, I think it is likely I can finish most sprints a day or two early and invest that remaining time into report writing and completing other college work. This should in turn free up the schedule of future sprints a bit more to allow for more project focus.

Besides these, the only other problem faced was trying to find relevant sources for research. However this will not be a reoccurring point during implementation as research is completed.

\section{Conclusion}
\label{section:conclusion}
% Enumerate the main conclusions you have got in terms of background, problem description and the solution approach you have come up with.

With the research phase of this project now complete, there is now a good overview of the problem being tackled along with the proposed solution.

As stated at the beginning of this report the initial problem statement came to me during my time working in Teamwork where the HR department had several issues related to event organisation with large numbers of attendees.

While this project does not aim to tackle only these specific issues, the initial idea grew into the project as it is today which takes a slightly broader approach in addressing the problem. in effect meaning this is a more generic solution rather than a Teamwork specific solution.

Given the problems often associated with event management on a large scale, the proposed system will help to tackle many organisation related issues as this is where the majority of problems stem from.

\section{Future Work}
\label{section:futurework}
% Enumerate all the things you would have wanted to do should you have more time to work on this report.

In regard to further developing this report I briefly considered including a desktop application and its associated research in this project. However, I eventually decided that such an extra, major piece added to this proposed system would be unlikely to be completed and only further strain already limited time. I believe this would have been a great addition to this project but in hindsight I believe I made the right choice in leaving it out as I wouldn't have had the time to research it, and likely would not have the time either to develop it in the project implementation phase. 

I also wanted this to be a mobile-first project, as this is the way in which users are most increasingly accessing online services, and therefore a mobile-first approach lends to allowing greater access to the app being developed. Given this point I further felt that attempting to add in a desktop application would take time away from making the mobile app as well made as possible.

Aside from the desktop application I also very briefly considered a web-based application but I am very inexperienced in web development, having only a passing knowledge of some server side development. Any attempt to develop a frontend web app would most likely end with a poorly made product and again, time taken away from the mobile app thus lessening what can be accomplished there.

% Additional resources on the use of latex is below.

% Tutorials:
% \begin{itemize}
%     \item \url{https://www.latex-tutorial.com/tutorials/beginners/how-to-use-latex}
%     \item \url{https://en.wikibooks.org/wiki/LaTeX}
%     \item \url{https://www.sharelatex.com/learn/Main_Page}
%     \item \url{http://www.math.harvard.edu/texman}
%     \item \url{https://web.stevens.edu/hfslwiki/images/a/a0/ShareLatex_Tutorial.pdf}
% \end{itemize}

% Presentations:
% \begin{itemize}
%     \item \url{http://www.iu.hio.no/~frodes/rm/ppt/latex.ppt}
%     \item \url{https://classes.soe.ucsc.edu/ams200/Fall09/Latex_intro.ppt}
%     \item \url{http://www.menet.umn.edu/~blake/latexcourse/courseslides.ppt}
% \end{itemize}
